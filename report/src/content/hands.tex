\section{Sterilisation}

\subsection{Kurzbeschreibung des Experiments}
Probanden legen Handschuhe an auf welche diese ein Kontrastmittel in Form
einer Creme einreiben. Die mit Kontrastmittel eingeriebenen Hände stellen
in diesem Versuch kontaminierte Hände das eine beliebigen Alltagssituation.
Die Hände werden in diesem Zustand in einer Lichtabschottenden Kammer
betrachtet welche im innern mit einer UV-Lampe bestückt ist. Das auf den
Händen aufgetragene Kontrastmittel reagiert mit dem UV-Licht und wird für
den Betrachtet von blossem Auge sichtbar.

In einem weiteren Schritt werden die Hände wie gewohnt gewaschen mit Wasser
und Seife. Nach dem Waschgang werden die Hände wieder unter dem UV-Licht
betrachtet.

\subsection{Hypothese}
Das gewöhnliche Händewaschen beseitigt den Grossteil des Kontrastmittels
von den Händen.

\subsection{Ergebnis}
Die Betrachtung unter dem UV-Licht zeigte vor dem Händewaschen eine
gleichmässige kontamination der Hände. Die Betrachtung nach dem 
gewöhnlichen Waschgang zeigte auf, dass auf gewissen Stellen, wie etwa
der Handinnenfläche oder dem Handrücken, nahezu kein Kontrastmittel mehr
festgestellt werden konnte. Deutliche Rückstände zeigten sich an den
dafür typischen Stellen, wie etwa den Fingerzwischenräumen und der
Aussenseite der Daumen.

\subsection{Fazit}
Das gründliche Händewaschen, wie es in Spitälern und ähnlichen Anlagen
vorgesehen ist, muss berücksichtigt und praktiziert werden um 
Kontaminationen über die Hände zu minimieren.
