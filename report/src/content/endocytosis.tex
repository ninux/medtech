\section{Endozytose}

\subsection{Kurzbeschreibung des Experiments}
Die Nahrungsaufnahme eines Einzellers ist zu untersuchen.
Hierzu wird rot eingefärbte Hefe eingesetzt als Nahrung und Pantoffeltierchen
als Einzeller.

\subsection{Hypothese}
Die rot eingefärbte Hefe wird sich im innern des Einzellers blau färben
bedingt durch deren Verdauungsprozess.

\subsection{Beobachtung}
Die Beobachtung unter dem Lichtmikroskop zeigt, dass sich die Einzeller
verschieden aktiv bewegen. Einige verharren in bestimmten Regionen und
andere bewegen sich eher rasch durch die gesamte Lösung. 

Die Nahrungsaufnahme kann nicht beobachtet werden, weder bei den stationären
noch bei den dynamischen Einzellern. Deutlich erkennbar sind die Ansammlungen
der rot eingefärbten Hefe im innern der Einzeller. Eine blaue Färbung ist nicht
erkennbar. Bei einigen Exemplaren der Einzeller sind dunklere Ansammlungen
der eingefärbten Hefe zu erkennen. Ob diese Farbdifferenz durch die Verdauung
verursacht ist oder durch eine dichtere Ansammlung der Hefe kann nicht
beurteilt werden.

\subsection{Fazit}
Die Beobachtungen welche aus dem Experiment hervorgehen können die Hypothese
nicht bestätigen, dass sich die Hefe blau färbt durch die Verdauungsprozesse
der Einzeller.

Eine erneute Untersuchung mit stärker definierten Parametern
(z.B. begrenzter Bewegungsraum der Einzeller, höhere Viskosität der
Testlösung, niedrigere Dichte der Nahrung in der Lösung, Anzhal der Einzeller)
und längeren Untersuchungszeiten ist zu empfehlen.
