\section{Blutzucker}

\subsection{Kurzbeschreibung des Experiments}
Zwei Probandengruppen nehmen vier verschiedene Nahrungsmittel ein.
Vor der Nahrungszufuhr wird eine Blutzuckermessung erstellt. Die Messung
wird alle 10 Minuten nach der Einnahme wiederholt. 
\begin{table}[h!]
	\centering
	\begin{tabular}{l l}
		Nahrungsmittel & Dosierung \\
		\hline
		Wasser & ein Glas à 3dl \\
		Assugrin & acht Einheiten à 55mg aufgelöst in 1.5dl Wasser \\
		Zucker & sechs Einheiten à 4g aufgelöst in 1.5dl Wasser \\
		Snickers & ein Riegel à 57g
	\end{tabular}
	\caption{Liste der verwendeten Nahrungsmittel}
\end{table}

\subsection{Hypothesen und Modellannahmen}
Für das vorliegende Experiment wird davon ausgegangen, dass alle Probanden
gesund sind und einen normalen Metabolismus aufweisen. Hierzu gehört
insbesondere die Annahme, dass kein Proband Diabetes hat.

Die Hypothese für das Experiment besteht aus vier Aussagen:
\begin{enumerate}[label=(\alph*)]
	\item Die Blutzuckerwerte von Probanden mit gleicher
		Nahrungsmittelaufnahme korrelieren stark.
	\item Die Blutzuckerwerte von Probanden mit Wasser-Einnahme sind
		konstant.
	\item Die Blutzuckerwerte von Probanden mit Assugrin-Einnahme
		sind fallend.
	\item Die Blutzuckerwerte von Pronanden mit Zucker- und
		Snickers-Einnahme sind steigend und korrelieren stark.
\end{enumerate}

\subsection{Daten}
% \pgfplotstabletypeset{data/blood_01.txt}

\newpage
\subsubsection{Gruppe A}
\begin{figure}[h!]
	\centering
	\begin{subfigure}{0.45\textwidth}
		\begin{tikzpicture}
			\begin{axis}[
				title={Wasser},
				xlabel={Zeit [min]},
				ylabel={Blutzucker [mmol/l]},
				width=1\textwidth,
				xmin=-5,
				xmax=35,
				ymin=3,
				ymax=9,
				ytick={3.0,3.5,4.0,4.5,5.0,5.5,6.0,6.5,7.0,7.5,8.0,8.5,9.0}
			]
				\addplot table [y=$W_1$, x=t]{data/blood_01.txt};
				\addlegendentry{W 1a};
				\addplot table [y=$W_2$, x=t]{data/blood_01.txt};
				\addlegendentry{W 2a};
				\addplot table [y=$W_3$, x=t]{data/blood_01.txt};
				\addlegendentry{W 3a};
			\end{axis}
		\end{tikzpicture}
	\end{subfigure}
	\hfill{}
	\begin{subfigure}{0.45\textwidth}
		\begin{tikzpicture}
			\begin{axis}[
				title={Assugrin},
				xlabel={Zeit [min]},
				ylabel={Blutzucker [mmol/l]},
				width=1\textwidth,
				xmin=-5,
				xmax=35,
				ymin=3,
				ymax=9,
				ytick={3.0,3.5,4.0,4.5,5.0,5.5,6.0,6.5,7.0,7.5,8.0,8.5,9.0}
			]
				\addplot table [y=$A_1$, x=t]{data/blood_01.txt};
				\addlegendentry{A 1a};
				\addplot table [y=$A_2$, x=t]{data/blood_01.txt};
				\addlegendentry{A 2a};
			\end{axis}
		\end{tikzpicture}
	\end{subfigure}

	\begin{subfigure}{0.45\textwidth}
		\begin{tikzpicture}
			\begin{axis}[
				title={Zucker},
				xlabel={Zeit [min]},
				ylabel={Blutzucker [mmol/l]},
				width=1\textwidth,
				xmin=-5,
				xmax=35,
				ymin=3,
				ymax=9,
				ytick={3.0,3.5,4.0,4.5,5.0,5.5,6.0,6.5,7.0,7.5,8.0,8.5,9.0}
			]
				\addplot table [y=$Z_1$, x=t]{data/blood_01.txt};
				\addlegendentry{Z 1a};
				\addplot table [y=$Z_2$, x=t]{data/blood_01.txt};
				\addlegendentry{Z 2a};
			\end{axis}
		\end{tikzpicture}
	\end{subfigure}
	\hfill{}
	\begin{subfigure}{0.45\textwidth}
		\begin{tikzpicture}
			\begin{axis}[
				title={Snickers},
				xlabel={Zeit [min]},
				ylabel={Blutzucker [mmol/l]},
				width=1\textwidth,
				xmin=-5,
				xmax=35,
				ymin=3,
				ymax=9,
				ytick={3.0,3.5,4.0,4.5,5.0,5.5,6.0,6.5,7.0,7.5,8.0,8.5,9.0}
			]
				\addplot table [y=$S_1$, x=t]{data/blood_01.txt};
				\addlegendentry{S 1a};
				\addplot table [y=$S_2$, x=t]{data/blood_01.txt};
				\addlegendentry{S 2a};
			\end{axis}
		\end{tikzpicture}
	\end{subfigure}
\end{figure}

\newpage
\subsubsection{Gruppe B}
\begin{figure}[h!]
	\centering
	\begin{subfigure}{0.45\textwidth}
		\begin{tikzpicture}
			\begin{axis}[
				title={Wasser},
				xlabel={Zeit [min]},
				ylabel={Blutzucker [mmol/l]},
				width=1\textwidth,
				xmin=-5,
				xmax=35,
				ymin=3,
				ymax=9,
				ytick={3.0,3.5,4.0,4.5,5.0,5.5,6.0,6.5,7.0,7.5,8.0,8.5,9.0}
			]
				\addplot table [y=$W_1$, x=t]{data/blood_02.txt};
				\addlegendentry{W 1b};
				\addplot table [y=$W_2$, x=t]{data/blood_02.txt};
				\addlegendentry{W 2b};
			\end{axis}
		\end{tikzpicture}
	\end{subfigure}
	\hfill{}
	\begin{subfigure}{0.45\textwidth}
		\begin{tikzpicture}
			\begin{axis}[
				title={Assugrin},
				xlabel={Zeit [min]},
				ylabel={Blutzucker [mmol/l]},
				width=1\textwidth,
				xmin=-5,
				xmax=35,
				ymin=3,
				ymax=9,
				ytick={3.0,3.5,4.0,4.5,5.0,5.5,6.0,6.5,7.0,7.5,8.0,8.5,9.0}
			]
				\addplot table [y=$A_1$, x=t]{data/blood_02.txt};
				\addlegendentry{A 1b};
				\addplot table [y=$A_2$, x=t]{data/blood_02.txt};
				\addlegendentry{A 2b};
			\end{axis}
		\end{tikzpicture}
	\end{subfigure}

	\begin{subfigure}{0.45\textwidth}
		\begin{tikzpicture}
			\begin{axis}[
				title={Zucker},
				xlabel={Zeit [min]},
				ylabel={Blutzucker [mmol/l]},
				width=1\textwidth,
				xmin=-5,
				xmax=35,
				ymin=3,
				ymax=9,
				ytick={3.0,3.5,4.0,4.5,5.0,5.5,6.0,6.5,7.0,7.5,8.0,8.5,9.0}
			]
				\addplot table [y=$Z_1$, x=t]{data/blood_02.txt};
				\addlegendentry{Z 1b};
				\addplot table [y=$Z_2$, x=t]{data/blood_02.txt};
				\addlegendentry{Z 2b};
				\addplot table [y=$Z_3$, x=t]{data/blood_02.txt};
				\addlegendentry{Z 3b};

			\end{axis}
		\end{tikzpicture}
	\end{subfigure}
	\hfill{}
	\begin{subfigure}{0.45\textwidth}
		\begin{tikzpicture}
			\begin{axis}[
				title={Snickers},
				xlabel={Zeit [min]},
				ylabel={Blutzucker [mmol/l]},
				width=1\textwidth,
				xmin=-5,
				xmax=35,
				ymin=3,
				ymax=9,
				ytick={3.0,3.5,4.0,4.5,5.0,5.5,6.0,6.5,7.0,7.5,8.0,8.5,9.0}
			]
				\addplot table [y=$S_1$, x=t]{data/blood_02.txt};
				\addlegendentry{S 1b};
				\addplot table [y=$S_2$, x=t]{data/blood_02.txt};
				\addlegendentry{S 2b};
				\addplot table [y=$S_3$, x=t]{data/blood_02.txt};
				\addlegendentry{S 3b};
			\end{axis}
		\end{tikzpicture}
	\end{subfigure}
\end{figure}

\newpage
\subsection{Ergebnis}
Die erfassten Daten werden hinsichtlich der formulierten Modellannahmen
und Hypothesen ausgewertet.

\subsubsection{Hypothese (a)}
Die Hypothese, dass Probanden mit gleicher Nahrungsmittelaufnahme eine
starke Korrelation der Blutzuckerverläufe aufweist, kann mit den
erfassten Daten bestätigt werden.

Dieses Ergebnis deutet darauf hin, dass die Probanden der jewiligen
Nahrungsmittelgruppe einen ähnlichen Metabolismus aufweisen und die
eingenommenen Nahrungsmittel eine einschlägige Wirkung auf den
Blutzuckerspiegel aufweisen.

\subsubsection{Hypothese (b)}
Die Hypothese, dass Probanden mit Wasser-Einnahme einen konstanten
Blutzuckerspiegel aufweisen, kann mit den erfassten Daten bestätigt
werden.

Dieses Ergebnis deutet darauf hin, dass Wasser in der eingenommenen
Menge (3dl) keine signifikante Auswirkung auf den Blutzuckerspiegel
hat. 

\subsubsection{Hypothese (c)}
Die Hypothese, dass Probanden mit Assugrin-Einnahme einen fallenden
Blutzuckerspiegel aufweisen, kann mit den erfassten Daten nicht
bestätigt werden.

Dieses Ergebnis deutet darauf hin, dass die Einnahme von Assugrin,
ähnlich wie die Einnahme von Wasser, keine signifikante Auswirkung
auf den Blutzuckerspiegel hat. Dieses Ergebnis ist plausibel, da
Assugrin ein gemisch aus synthetischen Süssstoffen ist (Saccharin
und Cyclamat) und von Personen mit Diabetes zum süssen verwendet
wird.

\subsubsection{Hypothese (d)}
Die Hypothese, dass Probanden mit Zucker- und Snickers-Einnahme
steigend sind und stark korrelieren, kann mit den erfassten Daten
bestätigt werden.

Die Blutzuckerverläufe der Probanden mit Zucker-Einnahme weist
eine signifikant höhere Steigung auf als jene der Probanden,
welche ein Snickers eingenommen hat. Dieses Resultat ist plausibel,
da in Wasser aufgelöster Zucker schneller in den Blutkreislauf
aufgenommen werden kann als der Zucker, welcher in einem Snickers
enthalten ist.
