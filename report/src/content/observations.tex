\section{Menschliche und tierische Proben unter dem Mikroskop}

\newpage
\subsection{Menschlicher Dünndarm}

\subsubsection{Proben}
\begin{table}[h!]
	\centering
	\begin{tabular}{l l}
		Bezeichnung	& Mammal Ileum \\
		Probe 		& 31-5226
	\end{tabular}
\end{table}

\subsubsection{Aufzeichnungen}
\begin{figure}[h!]
	\centering
	\begin{subfigure}[b]{0.3\textwidth}
		\includegraphics[width=1\textwidth]{../images/01_mammal_illeum.jpg}
		\caption{Objektiv 4x}
		\label{fig:01_mammal_ileum}
	\end{subfigure}
	\begin{subfigure}[b]{0.3\textwidth}
		\includegraphics[width=1\textwidth]{../images/02_mammal_illeum.jpg}
		\caption{Objektiv 10x}
		\label{fig:02_mammal_ileum}
	\end{subfigure}
	\begin{subfigure}[b]{0.3\textwidth}
		\includegraphics[width=1\textwidth]{../images/03_mammal_illeum.jpg}
		\caption{Objektiv 10x}
		\label{fig:03_mammal_ileum}
	\end{subfigure}

	\begin{subfigure}[b]{0.3\textwidth}
		\includegraphics[width=1\textwidth]{../images/04_mammal_illeum.jpg}
		\caption{Objektiv 20x}
		\label{fig:04_mammal_ileum}
	\end{subfigure}
	\begin{subfigure}[b]{0.3\textwidth}
		\includegraphics[angle=270, width=1\textwidth]{../images/05_mammal_illeum.jpg}
		\caption{Objektiv 40x}
		\label{fig:05_mammal_ileum}
	\end{subfigure}
	\begin{subfigure}[b]{0.3\textwidth}
		\includegraphics[angle=270, width=1\textwidth]{../images/06_mammal_illeum.jpg}
		\caption{Objektiv 40x}
		\label{fig:06_mammal_ileum}
	\end{subfigure}
	\caption{Aufzeichnungen der Lichtmikorskopischen Darstellungen der
		Dünndarmproben}
	\label{fig:mammal_ileum}
\end{figure}

\subsubsection{Kommentar}
Die Abbildungen \ref{fig:01_mammal_ileum} bis \ref{fig:06_mammal_ileum}
zeigen die auffällige Struktur der Schleimhaut des menschlichen Dünndarms.
Diese hat eine gefaltete Struktur, welche eine Zunahme der Gesamtoberfläche
zur Folge hat. Diese Faltungen oder auch Einsenkungen bilden die sogenannten
Lieberkühn-Drüsen \cite{wiki-leberkuehn-krypten}. Mit einer grösseren Fläche
der Schleimhaut kann die Reaktion mit dem durchgehenden Nahrungsbrei
effizienter erfolgen. Inbesondere die Abbildungen \ref{fig:05_mammal_ileum}
und \ref{fig:06_mammal_ileum} zeigen diese deutlich auf.

\newpage
\subsection{Menschliche Leber}

\subsubsection{Proben}
\begin{table}[h!]
	\centering
	\begin{tabular}{l l}
		Bezeichnung	& Human liver \\
		Probe 		& 31-5388
	\end{tabular}
\end{table}

\subsubsection{Aufzeichnungen}
\begin{figure}[h!]
	\centering
	\begin{subfigure}[b]{0.3\textwidth}
		\includegraphics[width=1\textwidth]{../images/01_human_liver.jpg}
		\caption{Objektiv 10x}
	\end{subfigure}
	\begin{subfigure}[b]{0.3\textwidth}
		\includegraphics[width=1\textwidth]{../images/04_human_liver.jpg}
		\caption{Objektiv 20x}
	\end{subfigure}
	\begin{subfigure}[b]{0.3\textwidth}
		\includegraphics[width=1\textwidth]{../images/05_human_liver.jpg}
		\caption{Objektiv 20x}
	\end{subfigure}

	\begin{subfigure}[b]{0.3\textwidth}
		\includegraphics[width=1\textwidth]{../images/07_human_liver.jpg}
		\caption{Vene -- Objektiv 40x}
	\end{subfigure}
	\begin{subfigure}[b]{0.3\textwidth}
		\includegraphics[width=1\textwidth]{../images/08_human_liver.jpg}
		\caption{Galle -- Objektiv 40x}
	\end{subfigure}
	\begin{subfigure}[b]{0.3\textwidth}
		\includegraphics[width=1\textwidth]{../images/10_human_liver.jpg}
		\caption{Arterie -- Objektiv 40x}
	\end{subfigure}
	\caption{Aufzeichnungen der Lichtmikorskopischen Darstellungen der
		menschlichen Leber}
\end{figure}

\subsubsection{Kommentar}


\newpage
\subsection{Menschliche Bauchspeicheldrüse}

\subsubsection{Proben}
\begin{table}[h!]
	\centering
	\begin{tabular}{l l}
		Bezeichnung	& Pankreas \\
		Probe 		& 31-5442
	\end{tabular}
\end{table}

\subsubsection{Aufzeichnungen}
\begin{figure}[h!]
	\centering
	\begin{subfigure}[b]{0.3\textwidth}
		\includegraphics[width=1\textwidth]{../images/01_pankreas.jpg}
		\caption{Objektiv 10x}
	\end{subfigure}
	\begin{subfigure}[b]{0.3\textwidth}
		\includegraphics[width=1\textwidth]{../images/02_pankreas.jpg}
		\caption{Objektiv 10x}
	\end{subfigure}

	\begin{subfigure}[b]{0.3\textwidth}
		\includegraphics[width=1\textwidth]{../images/03_pankreas.jpg}
		\caption{Objektiv 20x}
	\end{subfigure}
	\begin{subfigure}[b]{0.3\textwidth}
		\includegraphics[width=1\textwidth]{../images/04_pankreas.jpg}
		\caption{Objektiv 20x}
	\end{subfigure}
	\caption{Aufzeichnungen der Lichtmikorskopischen Darstellungen der
		menschlichen Bauchspeicheldrüse}
\end{figure}

\subsubsection{Kommentar}

\newpage
\subsection{Menschlicher Magen}

\subsubsection{Proben}
\begin{table}[h!]
	\centering
	\begin{tabular}{l l}
		Bezeichnung	& human stomach \\
		Probe 		& 31-5100
	\end{tabular}
\end{table}

\subsubsection{Aufzeichnungen}
\begin{figure}[h!]
	\centering
	\begin{subfigure}[b]{0.3\textwidth}
		\includegraphics[width=1\textwidth]{../images/01_stomach.jpg}
		\caption{Objektiv 4x}
	\end{subfigure}
	\begin{subfigure}[b]{0.3\textwidth}
		\includegraphics[width=1\textwidth]{../images/02_stomach.jpg}
		\caption{Objektiv 10x}
	\end{subfigure}
	\begin{subfigure}[b]{0.3\textwidth}
		\includegraphics[width=1\textwidth]{../images/03_stomach.jpg}
		\caption{Objektiv 20x}
	\end{subfigure}
	\caption{Aufzeichnungen der Lichtmikorskopischen Darstellungen de
		menschlichen Magens}
\end{figure}

\subsubsection{Kommentar}

\newpage
\subsection{Menschliche Aorta}

\subsubsection{Proben}
\begin{table}[h!]
	\centering
	\begin{tabular}{l l}
		Bezeichnung	& human aorta \\
		Probe 		& n.a.
	\end{tabular}
\end{table}

\subsubsection{Aufzeichnungen}
\begin{figure}[h!]
	\centering
		\includegraphics[angle=270, width=0.8\textwidth]{../images/01_aorta.jpg}
		\caption{Objektiv 10x}
	\caption{Aufzeichnungen der Lichtmikorskopischen Darstellungen der
		menschlichen Aorta}
\end{figure}

\subsubsection{Kommentar}

\newpage
\subsection{Blutvergleich von Mensch und Tier}

\subsubsection{Proben}
\begin{table}[h!]
	\centering
	\begin{tabular}{l l l}
		Gegenstand
			& Bezeichnung
			& Probe \\
		\hline
		Vogelblut
			& Bird Blood smear
			& 31-3134 \\
		Froschblut
			& Frog Blood smear
			& 31-3128 \\
		Menschliches Blut
			& Human Blood smear
			& 31-3152 \\
		Eigenes Blut
			& n.a.
			& n.a. \\
	\end{tabular}
\end{table}

\subsubsection{Aufzeichnungen}
\begin{figure}[h!]
	\centering
	\begin{subfigure}[b]{0.3\textwidth}
		\includegraphics[width=1\textwidth]{../images/02_bird_blood.jpg}
		\caption{Vogelblut -- Objektiv 100x}
	\end{subfigure}
	\begin{subfigure}[b]{0.3\textwidth}
		\includegraphics[width=1\textwidth]{../images/01_frog_blood.jpg}
		\caption{Vogelblut -- Objektiv 100x}
	\end{subfigure}

	\begin{subfigure}[b]{0.3\textwidth}
		\includegraphics[width=1\textwidth]{../images/01_human_blood.jpg}
		\caption{Vogelblut -- Objektiv 100x}
	\end{subfigure}
	\begin{subfigure}[b]{0.3\textwidth}
		\includegraphics[width=1\textwidth]{../images/02_own_blood.jpg}
		\caption{Vogelblut -- Objektiv 100x}
	\end{subfigure}
	\caption{Autzeichnungen der Lichtmikorskopischen Darstellungen der
		verschiedenen Blutproben}
\end{figure}

\subsubsection{Kommentar}
